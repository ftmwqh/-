\documentclass{article}
\usepackage{ctex}
\usepackage{graphicx}
\usepackage{amsmath}
\usepackage{indentfirst}
\usepackage{titlesec}
\usepackage{setspace}
\usepackage{subfigure}
\usepackage{caption}
\usepackage{float}
\usepackage{booktabs}
\usepackage{geometry}
\usepackage{multirow}
\geometry{left=1.2cm,right=1.2cm,top=2cm,bottom=2cm}
\title{\songti \zihao{2}\bfseries 拉曼光谱预习报告}
\titleformat*{\section}{\songti\zihao{4}\bfseries}
\titleformat*{\subsection}{\songti\zihao{5}\bfseries}
\renewcommand\thesection{\arabic{section}}
\author{王启骅 PB20020580}
\begin{document}
	\maketitle
	\section{实验目的}
	测量材料的拉曼光谱并进行分析。
	\section{实验原理}
	Raman散射:
       非弹性碰撞;方向改变且有能量交换;


Raman位移:Raman散射光与入射光频率差$ \Delta\nu $


Raman散射的两种跃迁能量差:
$ \Delta E=h(\nu_0-\Delta \nu) $生stokes线;强;基态分子多;
$ \Delta E=h(\nu_0+\Delta \nu) $产生反stokes线;弱;

\subsection{Raman光谱的特点}
不同物质其拉曼光谱是不同的,就像人的指纹,可以用于光谱表征;


拉曼位移\\
对不同物质:$ \Delta\nu $不同;\\
对同种物质:$ \Delta\nu $与入射光频率无关,只与分子能级结构有关;


表征振动-转动能级的特征物理量;定性与结构分析的依据。


斯托克斯线和反斯托克斯线对称分布于瑞利线两侧,通常是测stokes线
\begin{equation}
	\frac{I_{as}}{I_s}\propto T
\end{equation}


拉曼散射与分子所处的状态无关

\subsection{Raman光谱与红外光谱比较}
相似之处:
 两者都能提供分子振动频率的信息,对于一个给定的化学键,其红外吸收频率与拉曼位移相等,均代表第一振动能级的能量。
 
 
 不同之处:
a 红外光谱的入射光及检测光都是红外光,而拉曼光谱的入射光和散射光大多是可见光。拉曼效应为散射过程,拉曼光谱为散射光谱,红外光谱对应的是与某一吸收频率能量相等的(红外)光子被分子吸收,因而红外光谱是吸收光谱。
b 机理不同:从分子结构性质变化的角度看,拉曼散射过程来源于分子的诱导偶极矩,与分子极化率的变化相关。通常非极性分子及基团的振动导致分子变形,引起极化率的变化,是拉曼活性的。红外吸收过程与分子永久偶极矩的变化相关,一般极性分子及基团的振动引起永久偶极矩的变化,故通常是红外活性的。
c 制样技术不同:红外光谱制样复杂,拉曼光谱无需制样,可直接测试水溶液。


③ 两者间的联系
    可用经验规则来判断分子的红外或拉曼活性:
a 相互排斥规则:凡有对称中心的分子,若有拉曼活性,则红外是非活性的;若红外活性,则拉曼非活性。
b 相互允许规则:凡无对称中心的分子,大多数的分子,红外和拉曼都活性。
c 相互禁止规则:少数分子的振动,既非拉曼活性,又非红外活性。
   如:乙烯分子的扭曲振动,在红外和拉曼光谱中均观察不到该振动的谱带。

\section{实验仪器}
激光器:40MW半导体激光器   532nm,
最常用Ar激光器   488.0/514.5nm,
频率高,拉曼光强大


试样室:发射透镜
   使激光聚焦在样品上,
收集透镜
   使拉曼光聚焦在单色仪的入射狭缝。
   
   
单色仪:仪器心脏,
1个光栅,2个狭缝,
减少杂散收光


拉曼光谱仪(自搭)



单晶硅片,毛细管,载玻片



CCl4(液) , YVO4:Tm3+(粉末), YVO4:Er3+(粉末)


\section{实验内容}
1、了解Raman测试系统组成及要求。


2、学习使用拉曼谱仪并测量材料的拉曼光谱。


3、学习处理分析拉曼光谱数据。

\end{document}