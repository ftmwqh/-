\documentclass{article}
\usepackage{ctex}
\usepackage{graphicx}
\usepackage{amsmath}
\usepackage{indentfirst}
\usepackage{titlesec}
\usepackage{setspace}
\usepackage{subfigure}
\usepackage{caption}
\usepackage{float}
\usepackage{booktabs}
\usepackage{geometry}
\usepackage{multirow}
\geometry{left=1.2cm,right=1.2cm,top=2cm,bottom=2cm}
\title{\songti \zihao{2}\bfseries 红外荧光光谱预习报告}
\titleformat*{\section}{\songti\zihao{4}\bfseries}
\titleformat*{\subsection}{\songti\zihao{5}\bfseries}
\renewcommand\thesection{\arabic{section}}
\author{王启骅 PB20020580}
\begin{document}
	\maketitle
	\section{实验目的}
	测量红外荧光光谱
	\section{实验原理}
	荧光:光致发光的冷发光现象,是辐射跃迁的一种,是物质从高能态向低能态跃迁所释放的辐射。
	
	
	分子能级:
	
	
基态:一个分子中的所有电子的排布都遵从构造原理。


激发态:当一个分子中的电子排布不完全都遵从构造原理时,分子被称为处于激发态。


构造原理:能量最低原理;泡利不相容原理;洪特规则。


跃迁特点:


基态(S0)→激发态(S1、S2、激发态振动能级):
	吸收特定频率的辐射;
	量子化;
	跃迁一次到位;
	
	
激发态(S1、S2、激发态振动能级)→基态(S0) :
	多种途径和方式(见能级图);
	速度最快、激发态寿命最短的途径占优势;
	
	
	电子激发态的多重度:M=2S+1,
   S为电子自旋量子数的代数和(0或1);
平行自旋比成对自旋稳定(洪特规则),三重态能级比相应单重态能级低;大多数有机分子的基态处于单重态;


激发态停留时间短、返回速度快的途径,发生的几率大,发光强度相对大;
荧光:10-7~10 -9 s,第一激发单重态的最低振动能级→基态;
磷光:10-4~10s;第一激发三重态的最低振动能级→基态;
 
\subsection{辐射能量传递过程}
荧光发射:电子由第一激发单重态的最低振动能级→基态( 多为 S1→ S0跃迁),发射波长为$ \lambda_2' $荧光; $ 10^{-7}\sim 10^{-9} $s 。发射荧光的能量比分子吸收的能量小,波长长;  $ \lambda_2' $ > $ \lambda_2 $> $ \lambda_1 $;


 磷光发射:电子由第一激发三重态的最低振动能级→基态( T1 $\rightarrow$ S0跃迁):
         电子由S0进入T1的可能过程:( S0$\rightarrow$ T1禁阻跃迁)
 S0 $\rightarrow$激发$\rightarrow$振动弛豫$\rightarrow$内转移$\rightarrow$系间跨越$\rightarrow$振动弛豫$\rightarrow$ T1
         发光速度很慢: $ 10^{-4}\sim $100 s 。 
        光照停止后,可持续一段时间。
\subsection{光致发光}
概念:如果分子因吸收外来辐射的光子能量而被激    		发,所产生的发光现象称为光致发光。


光致发光的过程:


   当外部光源如紫外光,可见光甚至激光照射到光致发光材料时,发光材料就会发射出特征光如可见光,紫外光等,实际上光致发光材料的发光过程较复杂,一般由以下几个过程构成。
基质晶格或激活剂(或称激发中心)吸收激发能\\
基质晶格将吸收的激发能传递给激活剂\\
被激活的激活剂发出荧光而返回基态,同时伴随有部分非发光跃迁,能量以热的形式散发.
	\subsection{荧光激发光谱和发射光谱}
	任何因荧光化合物都具有两种特征光谱∶激发光谱和发射光谱。


           荧光激发光谱(激发光谱),就是通过测量荧光体的发光通量随波长变化而获得的光谱,它反映了不同激发光引起荧光的相对效率。激发光谱可供鉴别荧光物质,在进行荧光测定时供选择适宜的激发波长。
           
           
           荧光发射光谱又称荧光光谱,如果激发光的波长和强度保持不变,而让荧光物质所产生的荧光通过发射单色器后,照射于检测器上,扫描发射单色器并检测各种波长下相应的荧光强度,然后通过记录仪记录荧光强度对发射波长的的关系曲线,所得到的谱图,称为荧光光谱。
           
           
       荧光光谱表示在所发射的荧光中各种波长组分的相对强度。荧光光谱可供鉴别荧光物质,并作为在荧光测定时选择适当的测定波长或滤光片的根据。
       
       \subsection{稀土离子发光}
       稀土元素由于具有未充满的4f 电子壳层,且4f电子被外层的全满5s、5p电子壳层屏蔽,使稀土离子能带结构受环境的影响较小,使稀土离子具有类原子的光谱性质,容易发生能级跃迁,发射大量不同波长的光。
       
     
+3价稀土离子f-f跃迁呈现尖锐的线状光谱,发光的色纯度高。


荧光寿命跨越从纳秒到毫秒6个数量级。


吸收激发能量的能力强,转换效率高。


物理化学性质稳定,可承受大功率的电子束、高能辐射和强紫外光的作用。


\subsection{上转换荧光}
斯托克斯定律认为材料只能受到高能量的光激发,发出低能量的光。但是后来人们发现,其实有些材料可以实现与上述定律正好相反的发光效果,于是我们称其为反斯托克斯发光,又称上转换发光。


目前的主要应用为红外光激发发出可见光的红外探测,生物标识,和长余辉发光的警示标识,防火通道指示牌或者室内墙壁涂装充当夜灯的作用等
\subsection{光谱测试步骤}
1.粗测发射谱:根据吸收谱确定激发波长,或直接选用能量较高的蓝紫光作为激发波长,测试预期波长范围的发射光谱。


2.测试激发谱:根据发射谱确定发射波长,测试不同波长激发下该发射波长的荧光强度。


3.测试发射谱:根据激发谱确定最佳激发波长,测试发射谱。


\subsection{实验内容}

1、学习使用光谱测量尤其是弱光测量中常用的仪器设备。


2、测量 样品在 所需波段的激发光谱与发射光谱。


3、学习处理分析荧光光谱数据。

\subsection{实验设备与材料}

Horiba Fluorolog - 模块式荧光光谱仪、
 
 YVO4:Er3+,Yb3+

\subsection{实验方案}
方案一:

选择合适的参数测量YVO4:Er3+ 粉末样品的可见以及近红外激发光谱与发射光谱


方案二:(选做)

选用980 nm的激光器测试YVO4:Er3+粉末样品的上转换发光,根据选择合适的参数测量样品的可见波段的发射光谱


\end{document}